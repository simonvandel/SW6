\chapter{Collaboration in aSTEP}\label{ch:collab}

An important part of the project was the collaboration with the aSTEP core groups. This chapter describes the noteworthy cases of collaborations we have made with other groups of the aSTEP multi-project.

\section{Discussions On aSTEP}
We decided in the aSTEP multi-project to have weekly meetings. The purpose of the meetings was to have a forum for voicing concerns and requests with all aSTEP groups. At the start of the semester, during the first and second sprints, the agenda for each meeting mostly consisted of discussion about the architecture and specific API requirements, the latter being the most relevant for our project. Later in the semester the meeting mostly consisted of progress updates.

\subsection{API Requirements}
In Sprint 2, after we had elicited our initial system requirements, we had to inform the aSTEP platform groups about our requirements of the aSTEP platform. For example, we required that all position data available through aSTEP is assigned a specific ID corresponding to a person, however the ID should not make it possible to identify that person.

\subsection{Data Privacy}

During the third sprint, the main collaboration within the aSTEP multi-project was agreeing on the data privacy policy in aSTEP. There were two dominating opinions. The first were to make the aSTEP platform as open as possible, meaning data uploaded to the aSTEP platform is available to everyone, and the second to make the aSTEP platform access controlled, meaning data uploaded to the aSTEP platform is only available for a controlled number of users.

In the meetings we have had with SmukFest personnel during this project, a major concern was that the location data should not be publicly available. Because of this concern, it was agreed in the aSTEP project to make the aSTEP platform access controlled.


\section{Continuous Integration Maintenance}
To increase the overall quality of the software by the aSTEP groups, all aSTEP groups decided to use a continuous integration service to make sure that all software committed to the aSTEP projects pass a build test and unit tests.

We volunteered to setup and maintain the continuous integration service, which also included writing a guide for future use of the service.

\section{HTTPS Server Configuration}
Due to personal location data being handled by the aSTEP platform and in particular this project, we and the other members of the aSTEP project wanted to secure the network communication by encryption. As encryption algorithms are notoriously hard to get right, no non-expert should ever implement them themselves. We therefore decided to use HTTPS for all connections to and from the aSTEP platform. This ensures that no one can eavesdrop on sensitive data.

Since the user management group did not wish to prioritise the implementation of HTTPS, we decided to implement this part of the aSTEP platform ourselves, through a merge-request to the aSTEP project.
%The HTTPS support was added to the aSTEP core platform by the authors of this report, as we had prior experience setting this up.

\section{Forwarded Requirements}

As the aSTEP platform groups did not have direct contact with end users of the system, we helped them with their requirements by mediating the requirements, that was elicited from the interviews we conducted. The requirements was primarily forwarded with informal interviews conducted by the aSTEP platform groups.

Near the end of the multi-project we also assists some of the groups with evaluation of their services, through interviews about our usage of the aSTEP platform and our satisfaction with the API.

%sprint 2
%Collaboration with the aSTEP core during the sprint consisted of formulating requirement, that the core could use for developing the API calls that we require for retrieving data.


% hvordan vi delte use cases med andre fra aSTEP
%% evt. eksempler

% gitLab CI

% data sikkerhed
%% vi lavede pull request

% ugentlige møder

% terms of service
%% 

% aSTEP project focus discussion (Jens skriver)