\chapter{Discussion}

% intro
This chapter discusses considerations that have to be made before the system is put into use.

% ethics
\section{Ethics \& Malicious Usage}
Since this system is effectively able to display real-time positions of people, ethical issues and potential malicious usage have to be considered before the system should be used. 

If the system is receiving location data using Wi-Fi triangulation, positions of users can be recorded without user's consent. The tracking of users without their knowledge is a big ethical issue.

Malicious usage of the system can expose sensitive information about individuals. Although we do not receive information from the aSTEP project that can directly be used to identify individuals, it is possible to infer the identity of individuals using auxiliary information. For example, by combining knowledge about an individual's music taste, one can cross reference the location of all positions with specific music performances at a festival. Another potential issue in this system is finding individuals who appear to be alone in an area. 

% usefulness of the system
\section{Usefulness of The System}
%% pressure and turbulence was not intuitive. Is it even possible to make it intuitive? Can it get too high level, such that expert users can not use the system efficiently?
The evaluation of the system as described in \cref{sec:s4_test}, finds that the system is useful at visualising crowd factors, but only the velocity and density overlays were intuitive. One can argue whether the non-intuitiveness of the turbulence and pressure overlays is a fault of the system, or the concepts themselves. The turbulence and pressure concepts are not familiar to users as the density and velocity concepts, but on the other hand, there is room for additional research in other visualisation techniques.


%% data - router requirements (WIFI) OR gps data
The system is designed in a location based service-agnostic way such that multiple methods of capturing positions can be used. The aSTEP project supports positions captured by Wi-Fi triangulation and GPS. Both of these techniques have benefits and drawbacks.

Wi-Fi triangulation is useful as it does not require any user interaction. In order for the triangulation to work, many Wi-Fi access points have to be placed strategically on a premise to get the best accuracy. If this infrastructure is not already in place, it can be expensive to set up.

Positions gathered from users' GPS devices are more difficult to obtain as an application has to be installed on the user's device that continuously captures position data. This means that users have to interact with the system in order to be tracked. The accuracy of GPS positions can have varying quality depending on the environment the user is in.

%% accuracy of data
The usefulness of the system ultimately depends on the data the system operates on. The accuracy of the position data that is used in the system, has to be considered if the system is to be used.