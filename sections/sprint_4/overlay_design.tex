\section{Overlay Design} \label{sec:s4_overlay}
%%% Intro paragraph

\Cref{s3:analyser_implementation} describes how the analysis produces crowd factors for a set of points on a map. This section considers ways to visualise the result of the analysis in an intuitive way in accordance to the requirements S2-FR1 through S2-FR4 in \cref{tab:s3_req}, such that the information can be of help to an operator.

%%% Designing the different overlays
When visualising information, one has to be careful not to oversaturate the visualisation, which will cause information overload. We attempt to solve this problem by making each overlay toggleable, as described in \cref{sec:s2_design}. To keep each overlay simple, we will only use colour and opacity for conveying information. We have borrowed this idea from \citet{wirz2012inferring}.

Colour represents the value of the displayed crowd factor, e.g. the value for the density overlay could be two people per square metre. The colour scheme ranges from low values as blue to high values as red. What denotes a high value depends on the crowd factor being displayed, as explained in the \cref{subsec:levelsOfHazard}. \Cref{tab:color_scale} explains the colour scale. As different people associate different values to colours, it is difficult to find a universal colour scheme that is perfect for everyone. The chosen colour scheme is known as a 5 colour heatmap~\cite{fiveheatmap}, but this should be reconsidered according to each specific users needs, before being put into actual use.

\begin{table}[htbp]
\definecolor{low}{HTML}{0000FF}
\definecolor{medium}{HTML}{00FFFF}
\definecolor{high}{HTML}{00FF00}
\definecolor{higher}{HTML}{FFFF00}
\definecolor{highest}{HTML}{FF0000}
    \centering
    \begin{tabular}{l l l l l} \toprule
                        & Density & Velocity & Turbulence & Pressure \\ \midrule
    \cellcolor{low}     & 0       & 0        & 0          & 0         \\
    \cellcolor{medium}  & 1.775   & 1        & 0.25       & 0.0075         \\
    \cellcolor{high}    & 3.55    & 2        & 0.5        & 0.015         \\
    \cellcolor{higher}  & 5.325   & 3        & 0.75       & 0.0225         \\
    \cellcolor{highest} & 7.1     & 4        & 1          & 0.03         \\ \bottomrule
    \end{tabular}
    \caption[Crowd factor scale]{Table illustrating how the four crowd factors scale their colour. The max values follows the values presented in \cref{subsec:levelsOfHazard}. Since no high level of turbulence was found, it was chosen that the scale of turbulence from 0 to 1 should be represented by the complete colour scale. In \cref{subsec:levelsOfHazard} values of 0.04-0.045 were considered extremely high, and have been known to cause fatalities. We chose the value of 0.03 since we wanted the red colour to appear much earlier than this. The velocity levels were chosen to range from standing still to regular running speed. One could consider changing the colour scheme of the velocity arrows to be dependant on crowd flow and fitting this to the crowd flows found in \Cref{subsec:levelsOfHazard}.}
    \label{tab:color_scale}
\end{table}

Since the density of a crowd should often be considered when evaluating the danger represented by another crowd factor, we wish to convey the density levels at all times. This is done by making the opacity dependent on the density, an idea inspired by the work of \citet{wirz2012inferring}. The idea is that although an area has a high turbulence, the density of the area may be low, meaning that there is a lower probability of danger in the area. We define that the opacity can at maximum be 90\%, since the underlying map should always be visible.


Since each overlay is representing its values in the same manner, it supports a common intuition which helps the user better understand each crowd factor.

%%% Density, Pressure, Turbulence hexagoner
%%%% Hvordan vi gør det
\subsection{Density, Turbulence, and Pressure}
The three overlays density, turbulence, and pressure, have their visualisation technique in common. They render an analysed point as a hexagon. The arguments for using hexagons are the same as described in \cref{s3:select_points}.


%%% Velocity
\subsection{Velocity}
The velocity overlay require special attention since its value is a combination of the speed and direction of the crowd. We thought that an arrow would be the most intuitive way to show a direction and have its color represent the speed.

%%%% Første design ide med normale pile, problemer med dette

Our first design iteration used arrows with the shape as seen in \cref{fig:first_arrow_design}. However we found it difficult to determine the direction the arrow was pointing. Additionally we found that multiple arrows do not \enquote{compose} as well as we wanted to, as multiple arrows in succession did not give us a good indication of the general direction.



\begin{figure}[htbp]
\begin{subfigure}[c]{.49\linewidth}
    \centering
    \begin{tikzpicture} [
    hexa/.style= {shape=regular polygon,
                  regular polygon sides=6,
                  minimum size=1cm, draw,
                  inner sep=0,anchor=south,
                  fill=lightgray}
    ]

    \node[hexa] at (0,0) {\Large$\rightarrow$};
    \node[hexa] at (0,{1*sin(60)}) {\Large$\leftarrow$};
    \node[hexa] at ({1/2+1/4},{0.5*sin(60)}) {\Large$\rightarrow$};
\end{tikzpicture}
    \caption{First design iteration.}
    \label{fig:first_arrow_design}
\end{subfigure}
%
\begin{subfigure}[c]{.49\linewidth}
    \centering
    \begin{tikzpicture} [
    hexa/.style= {shape=regular polygon,
                  regular polygon sides=6,
                  minimum size=1cm, draw,
                  inner sep=0,anchor=south,
                  fill=lightgray}
    ]

    \node[hexa] at (0,0) {\Large$>$};
    \node[hexa] at (0,{1*sin(60)}) {\Large$<$};
    \node[hexa] at ({1/2+1/4},{0.5*sin(60)}) {\Large$>$};
\end{tikzpicture}
    \caption{Second design iteration.}
    \label{fig:second_arrow_design}
\end{subfigure}
\caption{Design iterations for arrows.}
\end{figure}


%%%% Andet design af pile (simpel), fordele

The second design iteration focused on improving the visibility of the arrowhead, and making the arrows compose better. \Cref{fig:second_arrow_design} illustrates this design. This design features arrows that are more reminiscent of the less-than and greater-than symbols. Because these arrows appear less symmetric, a break in the pattern is more noticeable. The  increased size of the arrowhead makes it easier to tell the direction.

The opacity in this overlay differs from the rest. Instead of the opacity being proportional to the density, it depends on the speed of the crowd. Density is still important when considering the danger level of the crowd velocity, but since the velocity overlay can easily be combined with the density overlay, opacity can be used in a different way. First, it does not make sense to display the direction of a crowd that moves at a slow pace, because there is no dominating direction. Second, as the surface area of the arrow is small, it can be hard to differentiate different opacity values. We therefore use a binary opacity; Either an arrow is shown, or it is not. The threshold value is defined as 0.18 meters per second based on non-scientific eye-balling tests. Finding a better value would likely happen in collaboration with users.

Since visualisation is a subjective matter, the considerations made in this section should also be viewed as such. Because of this we ought to evaluate our design in collaboration with potential users.