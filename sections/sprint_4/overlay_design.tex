\section{Crowd Factors Overlay Design} \label{sec:s4_overlay}
%%% Intro paragraph

\Cref{s3:analyser_implementation} describes how the analysis produces crowd factors for a set of points on a map. This section considers ways to visualise the result of the analysis, such that the information can be of help to an operator.

%%% Designing the different overlays
When visualising information, one has to be careful not to oversaturate the visualisation, which will cause information overload. We attempt to solve this problem by making each overlay toggleable, as described in \cref{}\sinote{indsæt note til sprint2 design}. To keep each overlay simple, we will only use color and opacity for conveying information. We have borrowed this idea from \citep{}\sinote{indæt franke}.

Color represents the value of the displayed crowd factor, e.g. the value for the density overlay could be 2 people per square metre. The color scheme ranges from low values as blue to high values as red. What denotes a high value depends on the crowd factor being displayed.

Since the density of a crowd should always be considered when evaluating the danger represented by another crowd factor, we wish to convey the density levels at all times. This is done by making the opacity depend on density. The idea is that although an area has a high turbulence, the density of the area may be low, meaning that there is less of a danger in the area.

The fact that each overlay is representing its values in the same manner, supports a common intuition which helps the user better understand each crowd factor.

%%% Density, Pressure, Turbulence hexagoner
%%%% Hvordan vi gør det
\subsection{Density, Turbulence, Pressure}
The three overlays density, turbulence, and pressure have their visualisation technique in common. They render an analysed point as a hexagon. The arguments for using hexagons are the same as described in \cref{s3:select_points}.

%%% Velocity
\subsection{Velocity}
The velocity overlay require special attention since its value is a combination of the speed and direction of the crowd. We thought that an arrow would be the most intuitive way to show a direction and have its color represent the speed.

%%%% Første design ide med normale pile, problemer med dette

Our first design iteration used arrows with the shape as seen in \cref{fig:first_arrow_design}. However we found it difficult to determine the direction the arrow was pointing. Additionally we found that multiple arrows do not \enquote{compose} as well as we wanted to, as multiple arrows in succession did not give us a good indication of the general direction.

\begin{figure}[htbp]
\begin{subfigure}[c]{.49\linewidth}
    \centering
    \begin{tikzpicture} [
    hexa/.style= {shape=regular polygon,
                  regular polygon sides=6,
                  minimum size=1cm, draw,
                  inner sep=0,anchor=south,
                  fill=lightgray}
    ]

    \node[hexa] at (0,0) {\Large$\rightarrow$};
    \node[hexa] at (0,{1*sin(60)}) {\Large$\leftarrow$};
    \node[hexa] at ({1/2+1/4},{0.5*sin(60)}) {\Large$\rightarrow$};
\end{tikzpicture}
    \caption{First design iteration.}
    \label{fig:first_arrow_design}
\end{subfigure}
%
\begin{subfigure}[c]{.49\linewidth}
    \centering
    \begin{tikzpicture} [
    hexa/.style= {shape=regular polygon,
                  regular polygon sides=6,
                  minimum size=1cm, draw,
                  inner sep=0,anchor=south,
                  fill=lightgray}
    ]

    \node[hexa] at (0,0) {\Large$>$};
    \node[hexa] at (0,{1*sin(60)}) {\Large$<$};
    \node[hexa] at ({1/2+1/4},{0.5*sin(60)}) {\Large$>$};
\end{tikzpicture}
    \caption{Second design iteration.}
    \label{fig:second_arrow_design}
\end{subfigure}
\caption{Design iterations for heading direction arrows.}
\end{figure}


%%%% Andet design af pile (simpel), fordele

The second design iteration focused on improving the visibility of the arrowhead, and making the arrows compose better. \Cref{fig:second_arrow_design} illustrates this design. This design features arrows that are more reminiscent of the less-than and greater-than symbols. Because these arrows appear less symmetric, a break in the pattern is more noticeable. The direction of the arrow is more explicit 

The opacity, ie. significance, of the arrows is binary. Either an arrow is shown, or it is not. If the general speed of the area is below a certain point, the significance of the heading direction is small, so we do not draw the arrow. Else, we draw the arrow.

%%% Summary
\sinote{Skal der være et summary? Kasper: yeah, probably}