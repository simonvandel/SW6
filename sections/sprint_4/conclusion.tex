\section{Conclusion of Sprint} \label{sec:s4_conclusion}
The fourth sprint started on April \nth{22}, and ended on May \nth{20}. In the sprint we continued the development of our application with the existing functional requirements, focusing on the overlays for visualisation of the four crowd factors. Non-functional requirements regarding maintainability and performance were also prioritised in this sprint.

To improve the maintainability of the system, the architecture of the model layer was aligned with the design decisions. The focus on performance is expressed by the implementation of asynchronous analysis.

To finalise the crowd factor overlays we considered how to visually represent the analysed data. Crowd density, turbulence, and pressure are visualised similarly, using colours to represent the value of the crowd factor, and opacity to represent the density. Velocity is visualised using arrows, marking the direction of the movement, where the colour of the arrow represents the speed of the movement. Contrary to the other overlays, the opacity of the arrows does not depend on the density, but the arrow is instead hidden in low velocity areas.

The sprint was concluded with a user acceptance test of the application, where the focus was on the usability and intuitiveness of the overlays. The test was only performed with a single user, and should as such not be considered definitive. The pressure overlay was not tested, as it was deemed too complex for the relatively short duration of the test.

While the overall feedback on the application was positive, it also presented some weaknesses of the application. Most notable of these were the lack of an intuitive understanding of the turbulence overlay, and the velocity arrows not being discernible. In order to improve the usability of the application, it would be necessary to consider how the visualisations can be made more intuitive.
