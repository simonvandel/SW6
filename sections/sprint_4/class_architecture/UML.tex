\subsection{Visualisation in UML}
In order to describe the structure design of the HotMap server, the UML v2.5 standard is used to express how the class hierarchy was designed. UML is a well known modelling language for visualising software system designs, in order to describe more easily the static structure of a software system. Therefore a UML class diagram is used for describing the model layer of the HotMap server.

Even though traits are not a common object oriented functionality like interfaces, UML has a way of representing a realisation of a trait. This is drawn by a dashed arrow from the class to the trait.

UML cannot represent mixin, so it was chosen to show this by drawing a realisation arrow with a mixin stereotype from the alias to the mixed type. Stereotypes are denoted by some text surrounded with \emph{<<} and \emph{>>} on top of the arrow. \jenote{teknisk set korrekt, men det kan godt forklares bedre}