\subsection{OOP in Scala}
Intro here

\subsubsection{Traits}
Scala does not have interfaces, but instead uses traits. These are similar but allow for default implementation of methods within the trait. When concrete classes extends traits, it is called a realisation of that trait, which in UML is indicated by a dashed arrow from the class to the interfaces. In Scala, traits can also be instantiated if all of its instance variables and methods have default implementations. Scala only allows traits to be mixed into classes or other traits.

\subsubsection{Mixin}
Mixin is a language dependent functionality that we use in our application, which allows a class \emph{A} to contain methods from other classes or interfaces without having \emph{A} be a specialisation of these other classes or traits. How class \emph{A} gain access to those methods through mixin is defined by the language. 

In Scala, type aliases of mixin types are used to denote a class or trait with a trait mixed into it. As an example of this we have on \cref{fig:class} that type \emph{Point} is an alias for any type which realises the trait \emph{Coordinate} mixed with the trait \emph{Weight}. UML cannot represent mixin, so it was chosen to show this by drawing a realisation arrow with a mixin stereotype from the alias to the mixed type. Stereotypes are denoted by some text surrounded with \emph{<<} and \emph{>>} above the arrow.

