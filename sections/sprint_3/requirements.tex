\section{Requirements Specification} \label{sec:s3_requirements}
During the user test with Alarm HS, several new use cases were identified. For the application to support these new use cases, additional requirements will be added to the previous requirements from \cref{sec:s2_reqs}.

\subsection{Revised Users} \label{ss:s3_users}
Along with the additional use cases, we also identified new users for the system. These users include guest users, medical users, and guard users.

The guest users are briefly described in \cref{tab:guest_user}. While guest users share some characteristics with the mobile users described in \cref{tab:mob_user}, they are specified as a separate user, as they should only have very limited access to the features of the system.

The medical and guard users are specialisations of the previously defined mobile user, and so references to mobile users will include both medical and guard users. A description of these users can be seen in \cref{tab:med_user,tab:guard_user}. These specific users are added in order to support targeted information in the system. This could be beneficial if there are incidents that only require the attention of one specific group.

\begin{table}[htbp]
    \centering
    \begin{tabularx}{\textwidth*3/4}{X}
        \toprule
        \multicolumn{1}{c}{\textbf{Guest user}} \\ 
        \midrule
        \textbf{Description:} A festival guest who moves around the festival. Their primary use of the system is reporting of incidents. \\
        \\
        \textbf{Characteristics:} There can be many guest users of the system at any time. As they mainly use the system for reporting incident, they will generally not have much experience in using the system. \\
        \bottomrule
    \end{tabularx}
    \caption{Description of the guest user.}
    \label{tab:guest_user}
\end{table}

\begin{table}[htbp]
    \centering
    \begin{tabularx}{\textwidth*3/4}{X}
        \toprule
        \multicolumn{1}{c}{\textbf{Guard user}} \\ 
        \midrule
        \textbf{Description:} A guard employed at the festival. They will mainly use the system to manually navigate around the area, and respond to incidents. \\
        \\
        \textbf{Characteristics:} There will be multiple guard users of the system, who will have varying experience with the system. \\
        \bottomrule
    \end{tabularx}
    \caption{Description of the guard user.}
    \label{tab:guard_user}
\end{table}

\begin{table}[htbp]
    \centering
    \begin{tabularx}{\textwidth*3/4}{X}
        \toprule
        \multicolumn{1}{c}{\textbf{Medical user}} \\ 
        \midrule
        \textbf{Description:} Medical personnel employed at the festival. They will primarily use the system for manual navigation when responding to medical incidents. \\
        \\
        \textbf{Characteristics:} There will be multiple medical users of the system. These users have varying experience with the system. \\
        \bottomrule
    \end{tabularx}
    \caption{Description of the medical user.}
    \label{tab:med_user}
\end{table}

\subsection{Additional Use Cases \label{ss:s3_uc}}
In addition to the previously identified use cases, during the user test some additional potential uses of the system were discussed. Based on the discussion we identified five additional use cases, described in \cref{tab:s3-uc1,tab:s3-uc2,tab:s3-uc3,tab:s3-uc4}.

\begin{table}[htbp]
    \centering
    \begin{tabularx}{\textwidth}{lX}
        \toprule
        \textbf{Name:}  & Mark incident on map \\ \midrule
        \textbf{ID:}    & S3-UC1 \\ \midrule
        \textbf{Description:} & A user marks an incident location on the map. \\ \midrule
        \textbf{Actors:} & Stationary users, mobile users \\ \midrule
        \textbf{Success scenario:} & 
        \begin{enumerate}
            \item The user marks a location on the map
            \item The user specifies the incident type
            \item The user specifies user groups that should respond to the incident
            \item The incident is marked on the map for relevant users
        \end{enumerate} \\ \bottomrule
    \end{tabularx}
    \caption{Mark incident on map use case description.}
    \label{tab:s3-uc1}
\end{table}

\begin{table}[htbp]
    \centering
    \begin{tabularx}{\textwidth}{lX}
        \toprule
        \textbf{Name:}  & Send notification \\ \midrule
        \textbf{ID:}    & S3-UC2 \\ \midrule
        \textbf{Description:} & A user sends a notification to other users. \\ \midrule
        \textbf{Actors:} & Stationary users, mobile users, guest users \\ \midrule
        \textbf{Success scenario:} & 
        \begin{enumerate}
            \item A stationary user access a notification menu
            \item The user writes a notification
            \item The user specifies the user groups that should receive this notification
            \item The user sends the notification
            \item All members of the specified user group receives the notification
        \end{enumerate} \\ \bottomrule
    \end{tabularx}
    \caption{Send notification use case description.}
    \label{tab:s3-uc2}
\end{table}

\begin{table}[htbp]
    \centering
    \begin{tabularx}{\textwidth}{lX}
        \toprule
        \textbf{Name:}  & Find contact information \\ \midrule
        \textbf{ID:}    & S3-UC3 \\ \midrule
        \textbf{Description:} & A user finds the contact information of another employee. \\ \midrule
        \textbf{Actors:} & Stationary users, mobile users \\ \midrule
        \textbf{Success scenario:} & 
        \begin{enumerate}
            \item The user enters the name of an employee as a search term
            \item The application finds the relevant employees
            \item The application shows the employees contact information
        \end{enumerate} \\  \bottomrule
    \end{tabularx}
    \caption{Find contact information use case description.}
    \label{tab:s3-uc3}
\end{table}

\begin{table}[htbp]
    \centering
    \begin{tabularx}{\textwidth}{lX}
        \toprule
        \textbf{Name:}  & Report incident on map \\ \midrule
        \textbf{ID:}    & S3-UC4 \\ \midrule
        \textbf{Description:} & A guest user reports an incident on the map. \\ \midrule
        \textbf{Actors:} & Guest users \\ \midrule
        \textbf{Success scenario:} & 
        \begin{enumerate}
            \item A guest user opens the application
            \item The user marks a location on the map
            \item The user describes the incident
            \item The incident report is sent to stationary and mobile users for confirmation
        \end{enumerate} \\  \bottomrule
    \end{tabularx}
    \caption{Report incident on map use case description.}
    \label{tab:s3-uc4}
\end{table}

\begin{table}[htbp]
    \centering
    \begin{tabularx}{\textwidth}{lX}
        \toprule
        \textbf{Name:}  & Visualise historical data \\ \midrule
        \textbf{ID:}    & S3-UC5 \\ \midrule
        \textbf{Description:} & Visualises previously recorded data on the map \\ \midrule
        \textbf{Actors:} & Stationary users \\ \midrule
        \textbf{Success scenario:} & 
        \begin{enumerate}
            \item The user requests historical data
            \item The historical data is visualised on the map
        \end{enumerate} \\ \bottomrule
    \end{tabularx}
    \caption{Visualise historical data use case description.}
    \label{tab:s3-uc5}
\end{table}

\subsection{Revised Requirements} \label{ss:s2_reqs}
Based on the new use cases, and the general feedback from the user test, several additional functional requirements have been identified, which must be added to the requirements previously identified in \cref{ss:s2_update_reqs}. The following requirements will be added:

\begin{enumerate}
    \item \emph{A map legend.} In order to make the crowd information overlays more intuitive, a map legend explaining the overlays must be added. This is very important for a usable system, and will be prioritised as must have.
    \item \emph{Multiple access levels.} The system must have multiple access levels, such that information and features can be targeted at the relevant users. This is important for the system, is prioritised as must have.
    \item \emph{Adding of markers at specific locations.} These markers could be used to contain information of incidents, and should be visible to all relevant users. This is prioritised as should have.
    \item \emph{A help feature.} The application should have a help feature that explains the applications functionalities. This is prioritised as should have.
    \item \emph{Send notification to other users.} It could be possible for some users with a sufficient access level to send notification to other users. This is prioritised as could have.
    \item \emph{A menu with searchable contact information.} A menu containing the contact information of the festival employees. It should be possible to search through the contacts. This is prioritised as won't have, as it is not important for the usability of the application.
    \item \emph{Reporting of incidents by festival guests.} It could be possible for festival guests to access the application, and report an incident on the map. This requirement is not in the scope of the project, and is prioritised as won't have.
\end{enumerate}

Updating the previous requirements specification from \cref{tab:s2_req} gives us the requirement specification shown in \cref{tab:s3_req}.

\begin{table}[htbp]
	\centering
	\begin{tabularx}{\textwidth}{lXl}
		\toprule
		\textbf{ID} & \textbf{Requirement} & \textbf{Priority} \\
		\midrule 
		\rowcolor[HTML]{EFEFEF} 
		S1-FR1 & A detailed map of the area & Must have \\
		S2-FR1 & A map overlay that intuitively visualises crowd density & Must have \\
		\rowcolor[HTML]{EFEFEF} 
		S2-FR2 & A map overlay that intuitively visualises crowd velocity & Must have \\
		S2-FR3 & A map overlay that intuitively visualises crowd turbulence & Must have \\
		\rowcolor[HTML]{EFEFEF} 
		S2-FR4 & A map overlay that intuitively visualises crowd pressure & Must have \\
		S1-FR3 & A graphical user interface for desktop computers & Must have \\
		\rowcolor[HTML]{EFEFEF} 
		S1-FR4 & A graphical user interface for smartphones & Must have \\
		S2-FR5 & Layers of details that can be toggled on and off & Must have \\
		\rowcolor[HTML]{EFEFEF} 
		S2-FR6 & User authentication & Must have \\
		S3-FR1 & A map legend & Must have \\
		\rowcolor[HTML]{EFEFEF}
		S3-FR2 & Multiple access levels & Must have \\
		S2-FR7 & A search function & Should have \\
		\rowcolor[HTML]{EFEFEF}
		S3-FR3 & Adding of markers at specific locations & Should have \\
		S3-FR4 & A help feature & Should have \\
		\rowcolor[HTML]{EFEFEF} 
		S2-FR8 & Navigation assistance & Could have \\
		S2-FR9 & Automatic warning system & Could have \\
		\rowcolor[HTML]{EFEFEF} 
		S2-FR10 & A visualisation of available personnel & Could have \\
		S3-FR5 & Send notification to other users & Could have \\
		\rowcolor[HTML]{EFEFEF}
		S2-FR11 & Projection onto camera images & Won't have \\
		S3-FR6 & A menu with searchable contact information & Won't have \\
		\rowcolor[HTML]{EFEFEF}
		S3-FR7 & Reporting of incidents by festival guests & Won't have \\
		\bottomrule
	\end{tabularx}
	\caption{Functional requirements for the third sprint.}
	\label{tab:s3_req}
\end{table}

No non-functional requirements are added in this sprint, so the non-functional requirements are still given by \cref{tab:s2_nreqs}.

\lanote{har ikke lige fundet ud af hvad jeg gør med det her krav endnu}
% Or is this a non-functional requirement?? TBD
\textbf{Should have}
\begin{enumerate}[resume]
    
    \item \textbf{[Added]} All data displayed to the user could also be stored for later retrieval, allowing for analysis of a problematic situation, which could improve the ability to handle similar future situations.
    
\end{enumerate}
