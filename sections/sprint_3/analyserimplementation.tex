\subsection{Implementation of the Analyser}\label{s3:analyser_implementation}

In order to fully understand the need for a good implementation of the analyser, we start with a quick rundown of the complexity of the unoptimized direct analysis algorithm seen in \cref{alg:unoptimised_algorithm}, which follows the mathematical formulas in \cref{sub:kernelDensityEstimation}.

\begin{center}
\captionof{algorithm}{Unoptimised algorithm}\label{alg:unoptimised_algorithm}
\begin{algorithmic}[1]

\Function{Analyser}{List[Point] points, List[person] People, Bandwidth}
\ForAll{points, x}
    \State KernelSum $\gets 0$
    \ForAll{People, P}
        \State KernelSum $\gets$ KernelSum $+$ \Call{Kernel}{P}
    \EndFor
    \State LocalDensity $\gets$ KernelSum $/$ (\Call{Count}{P} $\cdot$ Bandwidth)
\EndFor

\ForAll{points, x}
    \State KernelSum $\gets 0$
    \State VelocitySum $\gets 0$
    \ForAll{People, P}
        \State VelocitySum $\gets$ VelocitySum $+$ \Call{VelocityOf}{P} $\cdot$ \Call{Kernel}{P}
        \State KernelSum $\gets$ KernelSum $+$ \Call{Kernel}{P}
    \EndFor
    \State x.LocalVelocity $\gets$ VelocitySum $/$ KernelSum
\EndFor

\ForAll{points, x}
    \State KernelSum $\gets 0$
    \State TurbuSum $\gets 0$
    \ForAll{People, P}%
        \State TurbuSum $\gets$ TurbuSum $+$ \Call{DirectionOf}{P} $\cdot$ \Call{Kernel}{P}
        \State KernelSum $\gets$ KernelSum $+$ \Call{Kernel}{P}
    \EndFor
    \State LocalTurbulence $\gets$ TurbuSum $/$ KernelSum
\EndFor

\ForAll{points, x}
    \State KernelSum $\gets 0$
    \State PressSum $\gets 0$
    \ForAll{People, P}
        \State VelocityDif $\gets$ (\Call{VelocityOf}{P} $-$ x.LocalVelocity)
        \State PressSum $\gets$ PressSum $+$ \Call{Abs}{VelocityDif}$^2$ $\cdot$ \Call{Kernel}{P}
        \State KernelSum $\gets$ KernelSum $+$ \Call{Kernel}{P}
    \EndFor
    \State LocalPressure $\gets$ x.LocalDensity $\cdot$ (PressSum $/$ KernelSum)
\EndFor
\EndFunction
\end{algorithmic}
\end{center}

This algorithm has an asymptotic time complexity of $$O(\text{points} \cdot \text{people})$$
This might seem decent, but using this algorithm one will not scale the amount of points, but rather the size of the area requested for analysis. Assuming a square area, the amount of points would be quadratic in the size of the side of the area.
%Because of this and the requirement for real time analysis \cref{} we want to reduce this as much as possible.

\subsubsection{Reducing Complexity}

The first thing we need to reduce is the complexity. This would mean either the reduction of points or people.

Points can obviously be reduced, but not without also reducing the granularity of the result. But even if we reduce the number of points, we are not able to reduce the complexity. So if we want to reduce the complexity we need to reduce the amount of people iterated through. Considering that the kernel function will return zero for any person further away from the point than the bandwidth, we can reduce the complexity given that a reasonable bandwidth is used. This also relies on the reasonable assumption that people take up space, and that therefore we can only have a certain amount people inside a given bandwidth.

But before we can utilize this, we need a way to check if a person is further away than the bandwidth, without having to iterate through all the people. Luckily such a way exists in the form of R-trees. An R-tree is a data structure that takes $O(n)$ to build, and allows us to do range searches in $O(log(n))$ \cite{rtree}. If we start the analysis with building an r-tree of the people, each point would only have to query $O(log(n)$ people, giving us the complexity of $$O(\text{people} + \text{points} \cdot log(\text{people}))$$ which reduces to $$O(\text{points} \cdot log(\text{people}))$$.

\kanote{hvad er faktisk kompleksiteten af range search?}

\subsubsection{Parallel Computation}
The algorithm described in \cref{alg:revised_algorithm} is embarrassingly parallel, each point in the algorithm can be calculated independently of one another. The calculation can be sped up by computing the calls to \emph{analysePoint} on multiple computational units.

\subsubsection{Reducing Constants}

Having reduced the complexity, we still need to reduce the amount of constants in our algorithm's run time. We wish to reduce the amount of iterations on the points, as well repeated calculations such as the Kernel values.

%% Hvorfor det er nice

\kanote{kommentér koden}
\begin{center}
\captionof{algorithm}{Optimised algorithm using RTree range search}\label{alg:revised_algorithm}
\begin{algorithmic}[1]

\Function{Analyser}{List[Point] points, rTree[person] rTree, Bandwidth}
\ForAll{points, p}
    \LineComment{We reduce the amount of people using a range search}
    \State CulledPeople $\gets$ \Call{rTree.inRange}{Bandwidth,p} 
    \State \Call{analysePoint}{p,CulledPeople}
\EndFor
\EndFunction

\Function{analysePoint}{Point x, List[person] people}

%kernelValSum, kernelValForPerson, nonNormVel, nonNormTur, turKernelValSum

\State KernelValueSum $\gets 0$
\State KernelValForPerson $\gets List[(0,vector(0.0,0.0))]$ \Comment{List of tuples containing individual Kernel values and velocity}
\State NonNormalisedVelocitySum $\gets vector(0.0,0.0)$
\State NonNormalisedTurbulenceSum $\gets 0$
\State TurbulenceKernelValueSum $\gets 0$ \Comment{Summarised Kernel variable adjusted for turbulence}


\ForAll{People, P}
    \State TempKernelValue $\gets$ \Call{Kernel}{P}
    \State KernelValueSum $\gets$ KernelValueSum + TempKernelValue
    \State KernelValForPerson $\gets$ \Call{Concat}{(P.Velocity,TempKernelValue),NonNormalisedVelocitySum}
    \State NonNormalisedVelocitySum $\gets$ NonNormalisedVelocitySum + Velocity * TempKernelValue
    \If{P.HeadingDirection is valid}
        \State NonNormalisedTurbulenceSum $\gets$ NonNormalisedTurbulenceSum + P.HeadingDirection * TempKernelValue
        \State TurbulenceKernelValueSum $\gets$ TurbulenceKernelValueSum + TempKernelValue
    \EndIf
\EndFor

%%%%%%%%%%%%%%%%%%%%%%%%%%%%%%%%%%%%%

\ForAll{points, x}
    \State KernelSum $\gets 0$
    \State VelocitySum $\gets 0$
    \ForAll{People, P}
        \State VelocitySum $\gets$ VelocitySum $+$ \Call{VelocityOf}{P} $\cdot$ \Call{Kernel}{P}
        \State KernelSum $\gets$ KernelSum $+$ \Call{Kernel}{P}
    \EndFor
    \State x.LocalVelocity $\gets$ VelocitySum $/$ KernelSum
\EndFor

\ForAll{points, x}
    \State KernelSum $\gets 0$
    \State TurbuSum $\gets 0$
    \ForAll{People, P}%
        \State TurbuSum $\gets$ TurbuSum $+$ \Call{DirectionOf}{P} $\cdot$ \Call{Kernel}{P}
        \State KernelSum $\gets$ KernelSum $+$ \Call{Kernel}{P}
    \EndFor
    \State LocalTurbulence $\gets$ TurbuSum $/$ KernelSum
\EndFor

\ForAll{points, x}
    \State KernelSum $\gets 0$
    \State PressSum $\gets 0$
    \ForAll{People, P}
        \State VelocityDif $\gets$ (\Call{VelocityOf}{P} $-$ x.LocalVelocity)
        \State PressSum $\gets$ PressSum $+$ \Call{Abs}{VelocityDif}$^2$ $\cdot$ \Call{Kernel}{P}
        \State KernelSum $\gets$ KernelSum $+$ \Call{Kernel}{P}
    \EndFor
    \State LocalPressure $\gets$ x.LocalDensity $\cdot$ (PressSum $/$ KernelSum)
\EndFor
\EndFunction


\end{algorithmic}

\end{center}