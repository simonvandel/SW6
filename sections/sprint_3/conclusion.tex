\section{Conclusion of Sprint}
The third sprint started on March \nth{2} and ended on the April \nth{21}. The work presented in this sprint is primarily the development of the central component of the project, namely the analyser. Based upon the requirement that the system needs to be able to analyse the crowd conditions, we incorporated and adjusted methods of utilising position data to estimate crowd conditions. This entailed a deeper understanding of the nature of the crowd factors, and how the mathematical modelling represents it, which was required in order to tailor it to our requirements.

While our sources of inspiration have primarily done analysis on past events, our requirement for near real-time analysis prompted us to design our application around the analyser, in order to increase our computational output. This incentive was passed through to the implementation, which also had the primary concern of optimising the analysis for speed. The results of both the design and implementation are considered successful, as they both manage to considerably reduce the required time for analysis.

There was also a focus on altering the calculations for the crowd factors to fit the results found in other sources, in order to compare values found in the analyser with empirically found levels of hazard found in other research. This was also successful, as all factors calculated in the analyser is on relative form.

While the calculations are considered correct in the mathematical sense, we have yet to evaluate the analysis in a practical setting.