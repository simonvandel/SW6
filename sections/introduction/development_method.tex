\section{Development method} \label{sec:dev_method}
The aSTEP multi-project follows an agile development process, where the semester is divided into four sprints, expected to have different focus. As aSTEP is a newly started multi-project, the initial sprint is expected to be mainly concerned with finding relevant material and inspiration, finding users for the applications, and agreeing on practise's to be followed during the development. The second and third should focus on the design and implementation of the required functionality, while the last sprint should focus on testing of the system.

In order to facilitate inter-group collaboration, our project follows a similar development process. This development process also suits the nature of the project well, as the requirements are not clearly understood early in the project, and are expected to be subject to changes during the development process. 

To account for these changes we start by identifying a list of requirements for the system, including only the requirements that are well understood, and perceived as being essential. This list is then reevaluated in the beginning of each subsequent sprint, adding additional functional requirements, and refining the already formulated requirements. This method is used in combination with an evolutionary prototyping approach, with an initial prototype developed in the first sprint. In subsequent sprints this prototype is refined, improving existing functionality, and adding new functionality as required, based on the updated requirement specification. Each sprint concludes with an evaluation of the prototype at that time, in the form of a user test or performance test. This evaluation then forms the basis for the refinement of the requirements for the following sprint. 


\subsection{Continuous Integration}
The aSTEP project uses \emph{git} for version control via the GitLab repository manager. In an attempt to ensure high quality of the system, the project uses continuous integration, through built in functionality in GitLab... \sinote{note that something is missing}
