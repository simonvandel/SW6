\section{The aSTEP multi-project} \label{sec:astep}
% Goal with the aSTEP project
Due to this increasing interest in \gls{lbs}, Aalborg University has started the aSTEP multi-project. The aSTEP multi-project aims at facillitating development of \gls{lbs} applications by providing certain \gls{lbs} core functionalities in a core that can be used when developing an \gls{lbs} application. Additionally the aSTEP project also includes the development of some showcase \gls{lbs} applications.

\subsection{Multi-project collaboration}
% Split into core + app group
The aSTEP multi-project consists of 40 students, divided into ten groups. The groups are split into two super-groups, a core super-group and a application super-group. The core groups are responsible for the development of the common core functionalities, while the app groups are responsible for development of applications that utilize the core functionality, showing the posibilities and usages of the aSTEP core platform. 

% Multi-project collaboration
To organize the multi-project weekly meetings were held, with one member from each group attending. The purpose of these meetings were to agree on common practices, tools, and methods used across the aSTEP groups, and sharing of information relevant to all aSTEP groups, such as information about the progress and plans of the different groups. At the beginning of the project it was agreed that an agile development method would suit the nature of the project well, and so the semester was divided into four development sprints, each of which lasted about four weeks.

\subsection{aSTEP architecture}
% aSTEP architecture
In the beginning of the project the aSTEP groups agreed upon an architecture for the project, such that all groups would have a common understanding of the structure of the project. This also ensured that the application groups would have a common understanding of how the core should be used. The project follows a 3-tier architecture with a database layer, a business logic layer, and a client layer. 

The database layer is responsible for storing location data and retrieving stored location data upon request, and provides a data-access API that is used by the business layer. The business layer provides the logic for collecting location data, and provides a RESTful API for use by the client layer. The client layer can consist of many different client applications that utilise location data in different ways.

By using a 3-tier architecture it is possible to split the logic of the system into several components, ensuring a low-coupling and high cohesion. This facilitates working in many different groups that have separate responsibilities.