\section{Problem Definition} \label{sec:problem_def}
% Motivation for our specific problem
Our group belongs to the application super-group, and as such our focus is on developing an application that provides \gls{lbs} through the use of the aSTEP core platform. Due to the large application area of LBS, the problem domain \jenote*{had}{has} to be limited. 

\jenote{It was chosen to focus on}{An area of particular interest for our group is} using \gls{lbs} to monitor the behaviour of crowds at certain events, namely events that attract large crowds. While these \jenote*{the event cannot control anything}{events can control} the total number of people entering the premises, large crowds can still pose many different problems for the event organisers. This is, amongst other reasons, due to the \jenote{unpredictable}{some statement.. Since we actually also try to predict a bit} movements that happens in areas with large crowds~\cite{wirz2012inferring}. The problem is complicated further at some events, where it is not feasible for the security personnel to manually monitor the entire area at once. 

% Problem statement
Therefore a \gls{lbs} application might be able to assist in the monitoring task, by gathering information about the positions and movement of the individuals in the crowd, through their mobile devices. This leads us to the problem statement for this project:

\vspace{0.5 cm}
\begin{center}
	\textbf{\textit{How can we develop an application that, using aSTEP services, can analyse position data gathered from events that attracts large crowds, identify relevant information about the behaviour of the crowds, and present this information to event staff in an intuitive manner?}}
\end{center}