\section{Initial Requirements}\label{sec:s1_requirements}

%\begin{enumerate}
%    \item A mechanism for tracking crowds at an event using information collected via mobile technology.
%    \item A visualisation of crowds. The visualisation should aid event organisers in improving the experience at large events.
%\end{enumerate}

In this section we specify the initial requirements for the system, based on our problem statement. As we do not have a concrete user for the system, the initial requirements will be based on our assumptions about the problem domain. It is assumed that the system will be used both by a central stationary user, who needs an overview of the crowd, and by users moving around the event, for example when responding to an emergency. 

As the requirements are currently only based on our assumptions, they will most likely change during the development of the system. In order to better incorporate changes into the requirements, and to better suit the agile development method used, we prioritize the requirements using the MoSCoW method.

\lanote{source on MOSCOW}
This method groups the requirements into four categories; Must have, Should have, Could have, and Won't have. The requirements in the Must have category are considered essential to the system. If these requirements are not met by the system, the system can not be considered a success. The Should have category contains requirements that are very important to have in the system, but are often not as critical as the Must have requirements. Could have are requirements that are desirable, but not essential to the functionality of the system, though they may improve the usability of the system. Won't have cover requirements that the user would like to have, but they are not important to include in this release of the system.

The following requirements have been identified.

\textbf{Must have}
\begin{enumerate}
    \item The system must be able to analyse position data collected from the crowd via mobile technology.
    \item The system must be able to present a visualisation of relevant information about the state of the crowd.
    \item The system must be able to run on multiple platforms.
\end{enumerate}

In order to better understand the subject of crowd management, and how the current state of the art analyses and visualises crowds, related work is examined.

%Due to the need of collecting information about the crowd, and using this information to visualise the state of the crowd, some relevant work concerning these problems will be examined.