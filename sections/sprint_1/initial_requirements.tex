\section{Initial Requirements}\label{sec:s1_requirements}
\lanote{Ret igennem for application vs system.. + Overvej rename af usecases}
Before we begin the development of the application, we need to have a more well defined understanding of the concrete functionalities the application should include. In order to identify these concrete functionalities we first consider potential users of the application. After this we can identify potential use cases for the application, that we can finally use to specify the requirements for the application. As we do not currently have a concrete user for the system, the initial requirements will be based on our assumptions about the problem domain. 

\subsection{Users}
Before we consider how the application should be used, we will consider who the potential users of the application are. We assume that the application will have two different types of users, stationary users and mobile users, who will use the application for different purposes. A description of the stationary and mobile users can be seen in \cref{tab:stat_user,tab:mob_user}, respectively.

Stationary users could be personnel in charge of crowd management. Their tasks would include identifying dangerous situations, and taking the proper actions if one such situation occurs, such as dispatching of relevant personnel to an incident location.

\begin{table}[h!]
    \centering
    \begin{tabularx}{\textwidth*3/4}{X}
        \toprule
        \multicolumn{1}{c}{\textbf{Stationary user}} \\ 
        \midrule
        \textbf{Description:} A stationary user who access the application from a stationary location. Their primary use of the system is maintaining an overview of the state of the crowd during the event. \\
        \\
        \textbf{Characteristics:} There can be multiple stationary users of the application at the same time, with varying experience in using the application. \\
        \bottomrule
    \end{tabularx}
    \caption{Description of the stationary user.}
    \label{tab:stat_user}
\end{table}

Mobile users include personnel located around the event area, such as guards and medical personnel. Their tasks could include responding to reports of incidents, from both a central dispatcher and event attendees.

\begin{table}[h!]
    \centering
    \begin{tabularx}{\textwidth*3/4}{X}
        \toprule
        \multicolumn{1}{c}{\textbf{Mobile user}} \\ 
        \midrule
        \textbf{Description:} A mobile user of the system. The user moves around the event area, and must access the system from any given position. Their primary use of the system is to get a quick overview of the state of the crowd when moving from one point to another. \\
        \\
        \textbf{Characteristics:} There can be multiple mobile users of the system, with varying experience in using the application, and varying knowledge of the event area. \\
        \bottomrule
    \end{tabularx}
    \caption{Description of the mobile user.}
    \label{tab:mob_user}
\end{table}

\subsection{Use Cases} \label{ss:s1_uc}
To define the usage of the system, we identify potential use cases. Each of these use cases describe an interaction between an actor and the system. These use cases will be the basis of the functional requirements for the application.

First we consider potential use cases for the stationary users. The application should help the stationary users by providing them with an overview of the entire event area, such that potentially dangerous situations could be identified early, and swift actions can be taken. The primary use case for stationary users is described in \cref{tab:s1-uc1}.

\begin{table}[h!]
    \centering
    \begin{tabularx}{\textwidth}{|l|X|}
        \hline
        \textbf{Name:}  & Survey crowd \\ \hline
        \textbf{ID:}    & S1-UC1 \\ \hline
        \textbf{Description:} & A user surveys the crowd, looking for potentially dangerous situations. \\ \hline
        \textbf{Actors:} & Stationary users \\ \hline
        \textbf{Success scenario:} & 
        \begin{enumerate}
            \item The user opens the application on a desktop computer
            \item The system presents a visualization of the crowd's state on a map
            \item The user sees a potential problem developing
            \item The user takes appropriate actions
        \end{enumerate}
        \\ \hline
        \textbf{Extensions:} & No issues visible
        \begin{enumerate}
            \item The user opens the application on a desktop computer
            \item The system presents a visualization of the crowd's state on a map
            \item The user sees no problems with the crowds
            \item No actions are required
        \end{enumerate} 
        \\ \hline
    \end{tabularx}
    \caption{Survey crowd use case description.}
    \label{tab:s1-uc1}
\end{table}

Next we consider the mobile users. They are likely stationed around the events area, and must respond to incident reports from these positions. The application should assist these users in finding a path to the incident location, avoiding congested areas thus reducing their response time. This use case is described in \cref{tab:s1-uc2}.

\begin{table}[h!]
    \centering
    \begin{tabularx}{\textwidth}{|l|X|}
        \hline
        \textbf{Name:}  & Find fastest route \\ \hline
        \textbf{ID:}    & S1-UC2\\ \hline
        \textbf{Description:} & A user must find the fastest route to a incident location. \\ \hline
        \textbf{Actors:} & Mobile users \\ \hline
        \textbf{Success scenario:} & 
        \begin{enumerate}
            \item The user opens the application on a smartphone
            \item The user identifies own position on the map
            \item The user identifies the incident location on the map
            \item The user observes the state of the crowd on the map
            \item The user identifies congested areas
            \item The user identifies a route to the incident location, avoiding the congested areas
        \end{enumerate}
        \\
        \hline
    \end{tabularx}
    \caption{Find fastest route use case description.}
    \label{tab:s1-uc2}
\end{table}

\subsection{MoSCoW Mehtod}
As the requirements are currently only based on our assumed use cases, they will most likely change during the development of the application. In order to better incorporate changes into the requirements, and to better suit the agile development method used, we prioritize the requirements using the MoSCoW method~\cite{moscow}.

This method groups the requirements into four prioritisation categories; \emph{Must have}, \emph{Should have}, \emph{Could have}, and \emph{Won't have}. These categories are described in \cref{tab:moscow}.

\begin{table}[h!]
	\centering
	\begin{tabularx}{\textwidth}{lX}
		\toprule
		\textbf{Priority} & \textbf{Description} \\
		\midrule
		\rowcolor[HTML]{EFEFEF} 
		\textbf{M}ust have & Requirements that are considered essential to the system. If these requirements are not met by the system, the system can not be considered a success. \\
		\textbf{S}hould have & Requirements that are very important to have in the system, but are often not as critical as the Must have requirements. \\
		\rowcolor[HTML]{EFEFEF} 
		\textbf{C}ould have & Requirements that are desirable, but not essential to the functionality of the system, though they may improve the usability of the system. \\
		\textbf{W}on't have & Requirements that the user would like to have, but they are not important to include in this release of the system. \\
		\bottomrule
	\end{tabularx}
	\caption{Description of the four MoSCoW categories~\cite{moscow}.}
	\label{tab:moscow}
\end{table}

\subsection{Functional requirements}
Here we identify the functional requirements for our application, based on the two use cases in \cref{tab:s1-uc1,tab:s1-uc2}. 

First we consider the \emph{Survey crowd} use case shown in \cref{tab:s1-uc1}. We can see that the application must feature a detailed map of the event's area, along with a representation of the state of the crowd on this map. We also see that the application must be accessible on a desktop computer. From this use case we define the following requirements:

\begin{enumerate}
    \item \textbf{A detailed map of the area.} The application must feature a map of the event's area. The map must be detailed enough to facilitate identification of the user's own location and the reported incident location. The map details also help users plan their route to their destination. This requirement is essential to the application, and will as such be prioritised as \emph{Must have}.
    \item \textbf{An intuitive visual presentation of relevant crowd information.} The application must visually present any relevant crowd information in an intuitive manner. The user should be able to quickly gain an overview of the state of the crowd using this information, and use the knowledge to avoid congested areas that might increase the travel time. This is also an essential part of the application, and is prioritised as \emph{Must have}.
    \item \textbf{A graphical user interface for desktop computers.} The stationary users primarily use the system to maintain an overview of the event area from a stationary position. For this purpose the application should be running constantly during the event, and a large screen is preferable for giving a better overview. Therefore it is preferable for these users to access the application on a desktop computer. This requirement is also prioritised as \emph{Must have}.
\end{enumerate}

If we consider the \emph{Find fastest route} use case in \cref{tab:s1-uc2}, we see that this use case require many of the same features as the use case in \cref{tab:s1-uc1}. However, as the mobile users are moving around the event area, they must be able to access the application from any location from a smartphone. Therefore we add the following requirement.

\begin{enumerate}[resume]
    \item \textbf{A graphical user interface for smartphones.} As the user is mobile and moving around the area, the service provided by the application must be available to the user at any given location, accessible through the users smartphone. This requirement is essential, and is as such prioritised as \emph{Must have}.

\end{enumerate}

Combining these requirements we get the requirements specification shown in \cref{tab:s1_req}.

\begin{table}[h!]
	\centering
	\begin{tabularx}{\textwidth}{lXl}
		\toprule
		\textbf{ID} & \textbf{Requirement} & \textbf{Priority} \\
		\midrule 
		\rowcolor[HTML]{EFEFEF} 
		S1-FR1 & A detailed map of the area & Must have \\
		S1-FR2 & An intuitive visual presentation of relevant crowd information & Must have \\
		\rowcolor[HTML]{EFEFEF} 
		S1-FR3 & A graphical user interface for smartphones & Must have \\
		S1-FR4 & A graphical user interface for desktop computers & Must have \\
		\bottomrule
	\end{tabularx}
	\caption{MoSCoW prioritised functional requirements for the first sprint.}
	\label{tab:s1_req}
\end{table}

\subsection{Non-functional requirements}
In addition to the functional requirements, we can also defined some non-functional requirements for our application. Where the functional requirements specify the features that should be available to the users of the application, the non-functional requirements specify constraints for how the application should provide these features.

As the application is developed as a part of the aSTEP multi-project, the application should utilise the aSTEP core to provide we assume that future students might have to continue the development of the application.

\begin{table}[h!]
	\centering
	\begin{tabularx}{\textwidth}{lX}
		\toprule
		\textbf{ID} & \textbf{Requirement} \\
		\midrule 
		\rowcolor[HTML]{EFEFEF} 
		S1-NR1 & The application must use the aSTEP core functionalities to acquire location data \\
		\bottomrule
	\end{tabularx}
	\caption{Non-functional requirements identified in the first sprint.}
	\label{tab:s1_nreqs}
\end{table}

As the application must visually present relevant crowd information, we first examine related work, to identify the different kinds of information that should be displayed.