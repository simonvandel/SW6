\section{Conclusion of Sprint}
Our first sprint started on February 2 2016, and ended on February 26 2016. The focus of the sprint was to develop a functioning prototype, and establish contact with potential users for evaluating the prototype. A list of requirements was identified, after which work related to the problem domain was analysed. Based on the requirements and analysis, a prototype was developed. Contact was established with Alarm HS and Telia Parken, resulting in two interviews that included a general interview aimed at getting a better understanding of use cases for a solution, and user feedback on the prototype.

The prototype mainly fit the needs of Alarm HS, who could see many potential use cases for the program, and so Alarm HS will be considered the main user during the project. Their general feedback on the prototype was positive, and they liked the general idea of the application. They did have some ideas for improvements of the solution, which would be required for Alarm HS to be able to use it. This was expected as the initial prototype was based on assumptions of the use cases and requirements, as opposed to use cases and requirements elicited from real users. In the next sprint, the prototype will be developed further, in order to accommodate the requirements gained from Alarm HS.

During this sprint related work on the topic of crowd safety was also studied. This resulted in an insight into crowd conditions, and the hazards that can occur in crowds. From this four factors were found, that can be used as indicators of hazardous crowd conditions, and these will be further developed upon during the rest of the project.



%%%% START: Fit to report ...
%\lanote{Tilpas til sprint konklusion}
%Our first sprint started on February 2 2016, and ended on February 26 2016. The initial focus of the sprint was identifying relevant problems, and customers that could have an interest in an application that addresses such a problem. The problem in focus concerned crowd safety at festivals, where visualisation of the behavior of the crowds could possibly help alleviate pressure in certain areas. Contact was established with SmukFest, more specifically Alarm HS, whose main tasks concerns operational activities at the festival, such as dispatching help to festival guests. Additionally contact was made with the security officer Leif Bjørn in Telia Parken in Copenhagen, a stadium that hosts sporting events and concerts. It was established that visualization of the crowd would be more suitable for SmukFest, as the crowds at Telia Parken were easily observable and Leif Bjørn already had sufficient tooling. SmukFest is not as easily observable, due to the complex terrain at the festival and the larger area. 

%Different types of relevant information about crowd behavior were examined, and their relevance for the project evaluated. Finally a prototype was developed, displaying information assumed to be relevant for SmukFest. A meeting was held, where requirements for the application were discussed, and the prototype was presented and evaluated by two Alarm HS members. Feedback was given on the usability of the application, along with ideas for improvements and future development.

%Several meetings have been conducted with other groups in the aSTEP multi-project. The focus of the meetings has been on agreeing on a core architecture, and requirements for the CORE layer of aSTEP.
%%% END: Fit to report7