\section{Meetings} \label{sec:s1_meetings}
In order to find potential users and elicit requirements, different festivals and other potential users of crowd management software were contacted. This resulted in two meetings, one with the chief of security at Telia Parken and one with the operational leaders of Alarm HS, the internal control centre and dispatcher during the SmukFest festival. 

These meetings were held in the form of semi-structured interviews, consisting of a series of questions, along with a presentation of the prototype, with room for discussion. A list of questions was prepared, with the purpose of gaining a better understanding of the responsibilities of the security personnel when dealing with large crowds, along with identifying problems that arise in these scenarios. Additionally, the aim of the interviews was to gain an understanding of the technology currently used in their tasks.   

The following sections give an overview of the main points from the meetings.

\subsection{Telia Parken}\label{sub:telia_parken_meeting}
Telia Parken is Denmark's national stadium and the home field of FC Copenhagen, but many other events are also held there, such as concerts. They deal with a maximum of 70,000 attendants during their events, which requires a certain level of coordination from the security personnel. However the stadium is extensively covered by modern video surveillance, and due to the dividers in the stands, which completely separate the different sections of the stadium, as  well as the experienced personnel, they rarely have security incidents at the stadium.

The only way to improve the system according to the chief of security, Leif Bjørn, is to have a tool analysing crowd movement patterns and recognising dangers ahead of time, based on previous incidents. Such a system, although interesting, was deemed to be out of the scope of the project, and as such we had no further collaboration with Telia Parken.

However, the meeting revealed benefits of security in structural design, and the importance of understanding the mentality of a crowd. It also supported the idea that a very important aspect of crowd safety is access to information and the ability to understand that information.

The full summary can be seen in \cref{sec:telia_parken_meeting_summary}.


\subsection{Alarm HS}
Alarm HS is an internal service and command centre at SmukFest. Their tasks are to delegate incident reports ranging from injuries, illnesses, accidents, and violent behaviour, to more specialised teams. The team at Alarm HS is in direct contact with professional festival guards, the police and other emergency services. 

Currently, Alarm HS uses video surveillance to monitor the premises of the festival. The cameras are used to guide patrolling guards towards incidents, or areas of commotion. Other than the cameras, Alarm HS relies on the reports given by their guards for information.

The interviewed members of Alarm HS saw potential uses of the shown prototype, as it could give the operators an overview of the festival, which would supplement the more focused and precise information from the cameras. Another use could be to direct medical personnel through the festival areas, by avoiding crowded paths. The members of Alarm HS noted that the prototype lacked a display of buildings, bars and other fixpoints commonly used to locate positions throughout the festival areas. The Alarm HS members mentioned the need for buildings to be shown in the map, but as this wasn't supported by our prototype, we also presented our initial Google Maps based mock-up with a building overlay.

The representatives from Alarm HS were willing to collaborate with us and participate in further interviews and meetings, and will henceforth be considered the primary users.

The full summary can be seen in \cref{sec:first_meeting_alarmhs_summary}.