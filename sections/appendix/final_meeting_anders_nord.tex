\section{Final Meeting With Anders Nord}

The meeting was held on May 17, 2016 over Skype following a semi-structured form. Participating in the meeting was Anders Nord, substitute chairman of Alarm HS. The purpose of the meeting was to test the usability and performance of the system.

\begin{itemize}
    \item Anders is guided into the application.
    \item The tutorial scenario is shown to Anders.
    \item A small summary of which features has been added to the application.
    \item The density, velocity and turbulence overlays are examined in the tutorial, and explained.
    \item Anders is shown how he can reorder the z-order of the overlays.
    \item We switch to the test data scenario.
    \item Anders switches on the density overlay and can immediately tell that there is a high dense crowd in front of the stage, but that the surrounding areas look good. He also identifies an area about 50 meters west of the stage where the crowd is somewhat dense. He thinks that this point is a place where people from different directions are meeting.
    \item Anders notices a moving circle formation south of the stage. We suggest that he turns on the velocity overlay. He asks whether he can display the velocity overlay along with the density overlay, and we respond that he can.
    \item Anders did not notice that the velocity overlay displayed arrows at first, because they were very small. As he notices the arrows, he can see the crossroad of people moving towards each other.
    \item Anders suggests that he turns on the turbulence overlay to get a better view of the crossroad point. He now comments that he can see that the crossroad point has a high turbulence.
    \item He can explain that the turbulence is low in front of the stage because the crowd is standing still.
    \item Turning on the turbulence and the velocity overlays, he can see that groups moving in opposite directions are close to each other close to the festival entrance. There is turbulence where people are walking into each other.
    \item We talk about the accuracy of the system. The system visualisations are only a guide, not a fact.
    \item Anders likes the fence and fixpoints overlays, as it can give him more context to the map.
    \item Anders discusses the feature of adding the position of video cameras to the map, so he can faster switch to the correct video camera.
    \item We discuss the ability to track guards and medical staff.
    \item If a login could restrict the views a user can see, guards could also use the application.
    \item Anders mentions that it is hard for him to understand the different overlays and their use. It might be a good idea to train users before they use the system.
    \item He generally things the application is good at giving a overview of the entire festival area.
    \item We talk about the actual requirements for installation at the festival, and the limitations of the system e.g. dead spots.
    \item Anders is very interested in digitising Alarm HS's map, making it easier for them.
    \item 
\end{itemize}