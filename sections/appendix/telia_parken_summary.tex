\section{Meeting with Telia Parken}\label{sec:telia_parken_meeting_summary}

The meeting was held February 24, 2016 at Telia Parken, Copenhagen. Participating in the meeting was Leif Bjørn, who is head of security at Telia Parken.

\begin{itemize}
    \item His main task is to make sure that Parken is safe for all guests at all times, especially during events.
    \item Telia Parken hosts different types of events, mainly concerts and football matches, but also other events such as \emph{Top Charlie} events which includes dining guests and live music. 
    \begin{itemize}
        \item At concerts the guest are both standing on the field, and sitting on the stands. At these events the maximum capacity is 50,000 guests, of which 20,000 are stading on the field.
        \item At football matches all guests are on the stands.
        \item At Top Charlie events tables and benches are placed on the field for people to eat and drink at, and a dance floor is placed with live music. These events have around 4,500 guests.
    \end{itemize}
    \item Guards and inspectors are both stationary in key positions, and patrolling areas.
    \begin{itemize}
        \item Their tasks include observing the guests, and reporting to the central administration.
        \item Additionally they can calm down guests, force problematic guests to leave, and issue fines and bans.
        \item They are managed by Leif, communication via radio communication.
        \item They should not have an overview through a software system. They should only receive orders from Leif and act as observers for him.
    \end{itemize}
    \item During events, Leif is sitting in a control room, with police and fire department representatives, that has a good visual overview of the stadium.
    \begin{itemize}
        \item In the room video surveillance by 80 cameras is available. All cameras can be individually controlled.
        \item The video surveillance is both for general surveillance, but also for legal reasons. For instance, Parken has to be able to document why fines are given.
        \item The current system is stationary to this control room.
        \item The current system is a desktop computer. He prefers this over a mobile phone/tablet as it is more visually dominant, so that it catches his attention. He also uses his phone during the events and he would therefore not be able to use the system on his phone during phone calls.
    \end{itemize}
    \item The stands in Parken are separated into many sections, and it is not possible to go from one section to another. This layout helps Leif control the crowds, as he can have a precise count of people currently at the sections, which is very important for him.
    \item Their current system can visualise the current guests/capacity ratio for each section, and calculate the current flow into and out of the stadium. Everything in the current system is logged and can later be recalled.
    \item When shown our prototype, Leif liked the overview that it can give, but as the quality of the overview gives does not exceed their current system, he did not assess that Parken would be interested in the project.
    \item The interview was then directed to be more hypothetical (what if Parken did not have their current system?)
    \begin{itemize}
        \item He liked the idea of different overlays of information.
        \item He mentioned a concern of the system not being able to distinguish dangerous and non-dangerous turbulence. For instance, the fans of \emph{Football Club Copenhagen} has a tendency of standing up shoulder to shoulder and sway/moving from side to side while singing.
        \item He suggested that our proposed system would work better at festivals.
    \end{itemize}
    \item We were given a tour of the control room and their current system was showcased.
    \item During the meeting two potential directions for the project were discussed.
    \begin{itemize}
        \item Assist their current system. Since the system Leif and his personnel already use has a very quick response time with a good overview, the only usecase he could see was for analysis of the data. For example, he imagined a system in which he could tag episodes so that the system could analyse the patterns leading up to these events, and then give warnings when similar situations occur.
        \item Assume that they do not have such an extensive system and try to help accommodate the requirements for the security tasks, such a system needs to be able to give an overview of how close each stand is to being full, and show the densities outside of the stadium and how the entrances are pressured.
    \end{itemize}
\end{itemize}
