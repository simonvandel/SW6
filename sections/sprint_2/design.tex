\section{Design} \label{sec:s2_design}
This section covers the main design decisions made during this sprint. It first covers the choice of the architecture for the prototype, and then decisions regarding the user interface design.

\subsection{Web Application}
%During the user testing of the prototype in the first sprint it was confirmed that the application should be able to run on multiple platforms, which is also reflected in the requirement specification.\jenote{Måske bare referer til kravene} Therefore we consider two different approaches to meet this requirement. The first approach would consist of developing a separate application for each of the platforms. This would necessarily include applications for major operating systems for both desktop computers and mobile devices. The other approach is developing a web application. By doing this, any platform with a common browser will be able to use the application. This is advantageous as it means only one application has to be developed. Therefore we continue the project on the basis of a web application.\jenote{er vi efterhånden ikke kommet frem til det her en del gange?}

During the user testing of the prototype in the first sprint it was confirmed that the application should be able to run on multiple platforms. The prototype started out as a web application in the first sprint, which is a good choice, as web applications run on all platforms with a browser

\sinote{jeg har skrevet det om, hvad siger i?}

\subsection{Architecture} \label{sec:s2_architecture}
In \cref{sec:s1_prototype} we assumed that the system would be based on a client-server architecture, due to the need of supporting many users on different platforms simultaneously. This was confirmed during the meetings held with potential users at the end of the first sprint. As a concrete architecture was not crucial to the prototype developed during the first sprint, it was not formalised at that point. In the second sprint the architecture started to have a bigger impact on the development of our application, and as such we here present our considerations on the architecture.

The main reason for using the following architecture is to make the system more adaptable, which is enforced because of the separation between the functionality, the model layer and the representation layer. This also supports requirements S1-NR3 and S1-NR4 in \cref{tab:s2_nreqs}, of making the application maintainable and extensible.

We use the model-view-controller (MVC architectural pattern for the web application, as this pattern supports low-coupling and high cohesion)~\cite{website:MVC}. The aSTEP platform follows a client-server architecture, where the platform itself represents the server, and applications using the platform represents clients. As our application also follows a client-server architecture, the HotMap server will act as a middle-end between the aSTEP platform and the actual users. The architecture is shown in \cref{fig:architecture_diagram}.

\begin{figure}[htbp]
    \centering    
\tikzstyle{block} = [draw, fill=gray!20, rectangle, 
    minimum height=3em, minimum width=6em]
\tikzstyle{big} = [draw, fill=yellow!20, rectangle, 
    minimum height=6em, minimum width=14em]
\tikzstyle{big2} = [draw, fill=blue!20, rectangle, 
    minimum height=18em, minimum width=26em]
\tikzstyle{sum} = [draw, fill=blue!20, circle, node distance=1cm]
\tikzstyle{input} = [coordinate]
\tikzstyle{output} = [coordinate]
\tikzstyle{pinstyle} = [pin edge={to-,thin,black}]    
\tikzset{%
  cascaded/.style = {%
    general shadow = {%
      shadow scale = 1,
      shadow xshift = -1ex,
      shadow yshift = 1ex,
      draw,
      thick,
      fill = blue!20},
    general shadow = {%
      shadow scale = 1,
      shadow xshift = -.5ex,
      shadow yshift = .5ex,
      draw,
      thick,
      fill = gray!20},
    fill = gray!20, 
    draw,
    thick,
    minimum width = 1.5cm,
    minimum height = 1cm}}
    

\resizebox{\linewidth}{!}{
\begin{tikzpicture}[auto, node distance=4cm,>=latex']
    % We start by placing the blocks
    \node[] (center) {};
    \node[below of=center, node distance=2.5cm] (dummy) {};
    \node[big2, right of=dummy, node distance=3.4cm] (package) {};
    \node [block] (controller) {Controller};
    \node [block, below of=controller] (model) {Model};
    \node [below of=controller, node distance=4.5cm] (fake1) {};
    \node [big, right of=fake1, node distance=5.3cm] (viewGroup) {};
    \node[cascaded,label=below:, right of=fake1 ] (views) {};
    \node[block, right of=views, node distance=2.2cm] (json) {Json};
    \node[block, above of=controller] (user) {User};
    \node [cloud, draw,cloud puffs=10,cloud puff arc=120, aspect=2, inner ysep=1em, left of=controller, node distance=6cm] (astep)
    {aSTEP core};
    \node[below of=views, node distance=0.0cm] (viewText) {Views};
    \node[right of=controller, node distance=6.8cm] (hotmapDummy) {};
    \node[above of=hotmapDummy, node distance=0.5cm] (hotmap) {HotMap server};
    
    \draw (viewGroup) -- (5.2,4);


    \path[->]
            (controller)    edge  node[sloped, anchor=center, above, text width=2.0cm] { update view } (viewGroup)
            ([xshift=-1ex]controller.south)    edge  node[sloped, anchor=center, below, text width=2.0cm] { update model } ([xshift=-1ex]model.north)
            ([yshift=1ex]controller.west)    edge  node[sloped, anchor=center, above, text width=2.0cm] { aSTEP request } ([yshift=1ex]astep.east)
            (user)    edge  node[sloped, anchor=center, above, text width=2.0cm] { user request } (controller)
            (5.2,4)    edge  node[sloped, anchor=center, above, text width=2.0cm] { updated view } (user)
            ([yshift=-1ex]astep.east)    edge  node[sloped, anchor=center, below, text width=2.0cm] { aSTEP reply } ([yshift=-1ex]controller.west)
            ([xshift=1ex]model.north)    edge  node[sloped, anchor=center, below, text width=2.0cm] { response } ([xshift=1ex]controller.south);
            
\end{tikzpicture}

}
\caption{Architecture diagram}
\label{fig:architecture_diagram}
\end{figure}

This architecture describes how a request from a user is handled by the HotMap server. In this architecture the controller component serves as a request handler. The controller also handles computations such as analysing position data to determine crowd factors. The model components encapsulates information in classes for passing information between views and the controller easily. Information computed by the controller can either result in a view sent to the user, or data from the controller can be converted to JSON, and then update the data displayed on the user's current view. 

\jenote{You are going to hate me.. But i think that this is not descriptive and detailed enough, compared to what we were actually thinking at the time. At least not as far as our own application goes. We had ideas about the caches, misses and so on. If it is written here it will also be a let easier to reference later, and use as argumentation}


The following list describes the purpose of each transaction shown on \cref{fig:architecture_diagram} in the order they are evaluated.

\begin{enumerate}
    \item \textbf{user request:} The user requests a page at the HotMap server.
    \item \textbf{aSTEP request:} This transaction requests information needed for handling the request from the user. This can involve checking user credentials or collecting position information.
    \item \textbf{aSTEP reply:} The aSTEP server responds with the requested information.
    \item \textbf{update model:} The controller encapsulates the information requested by the user into models, or updates the state of the model layer as requested by the user.
    \item \textbf{model response:} The model layer responds to the controller with updated models.
    \item \textbf{update view:} This transaction sends information to the view layer, which then either updates the users view, or converts the information into Json.
    \item \textbf{user respond:} This transaction handles sending data to the user. It can either be a new view, or data to display on the current view.
\end{enumerate}

\subsection{Graphical User Interface Design} \label{sec:s2_gui}

During this sprint we designed a graphical user interface (GUI) for our web application. We based this design on the requirements we had from the meeting with the SmukFest personnel. The focus of the design was to maximise available information without causing information overflow, as well as making the application responsive, such that it automatically adapts to multiple screen sizes. 

The result is a map centric design, which can be seen in \cref{desktopscreenshot,phonescreenshot}, meaning that the map was considered the focus of the application, making it as visible as possible. The map manipulation options, such as overlays and zoom functions, were put on top of the map. Additional information, such as an employee information page, is accessible through a menu bar on a desktop computer, or on a slide-in menu on a phone or tablet screen, in accordance with the responsive design goal. A search bar, shown in \cref{desktopscreenshot,phonescreenshot} with the text \enquote{søg}, which searches through points of interest, is visible at all times on all devices. This was done in accordance with a feature wanted by the Alarm HS, as seen in requirement S2-FR7 in \cref{tab:s2_req}.

\begin{figure}[htbp]
    \centering
\includegraphics[width=\textwidth]{gui_desktop.png}
\caption{Screenshot of the application on a desktop computer.}
\label{desktopscreenshot}
\end{figure}
\begin{figure}[htbp]
    \centering
\includegraphics[width=\textwidth/3]{gui_phone.png}
\caption{Screenshot of the application on a smartphone.}
\label{phonescreenshot}
\end{figure}