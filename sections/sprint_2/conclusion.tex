\section{Conclusion of Sprint}
The second sprint started on February \nth{27}, and ended on March \nth{23}. The focus of the sprint was the further development of the prototype, in order to meet the requirements from the beginning of this sprint. During the sprint, a design was developed and implemented. The design resulted in an adjusted continuation of the design from the first sprint.

During the implementation process of the sprint, problems regarding geodetics was encountered, which prompted the need of choosing standards. The result was usage of the WGS84 standard for referencing, Web Mercator for projection, and Haversine for conversions.

The sprint was concluded with two meetings, both with personnel from SmukFest. The meeting with Søren Eskildsen from SmukFest provided insight into the system's possible use in a crowd management setting. Søren personally had some different requirements from what our first meeting with Alarm HS uncovered. Since the group focuses on the general management of the personnel during the festival in near real time, the group decided to continue the collaboration with Alarm HS instead of the crowd management group that Søren represents on the festival.

The second meeting with Anders from Alarm HS confirmed that the project was developing in the right direction. It also served to uncover additional areas where the system would need development.