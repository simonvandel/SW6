\section{Conclusion of Sprint}

The second sprint started on February 27, and ended on March 23. The focus of the sprint was the development of the prototype, in order to meet the requirements identified in the first sprint. During the sprint, a design was developed and implemented. The design was based on the requirements, and resulted in a adjusted continuation of the design from the first sprint.

During the implementation process of the sprint, problems regarding geodetics was encountered, which prompted the need of choosing standards. The result was usage of the WGS84 standard for referencing, Web Mercator for projection, and Haversine for conversions.

The sprint was concluded with two meetings, both with personnel from SmukFest. The meeting with Søren Eskildsen from SmukFest naturally lead to a delimitation of the system. Søren had some different requirements from what our first meeting with Alarm HS uncovered. Since the group focuses on the general management of the personnel during the festival in real time, the group decided to continue the collaboration with Alarm HS instead of the crowd management group that Søren represents on the festival.

The meeting with Anders from Alarm HS affirmed that the project was developing in the right direction. It also served to uncover additional areas where the system would need development.

Collaboration with the aSTEP core during the sprint consisted of formulating requirement, that the core could use for developing the API calls that we require for retrieving data.