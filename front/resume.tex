\chapter*{Resume}
%% Resume af rapporten (1-2 sider)

This report concerns itself with the domain of crowd safety. Specifically, the report documents the development of a software system for analysing crowd conditions, based on location data on users. This location data is gathered using the aSTEP platform. The system is targeted for crowd safety personnel working at large events such as music festivals. The report is divided into four sprints, each with a different focus.

In the first sprint, initial requirements for the system are based on assumptions about the problem domain. These requirements leads us to the concept of crowd conditions, which describes the state of a crowd in the context of crowd safety. Inspired by prior work, four crowd factors are found to be of interest to this project: Density, velocity, turbulence and pressure. In order to be able to showcase our work to interested parties early on in the project, we develop a prototype. The primary focus of this prototype is on exploring functionalities and usability of the system. This prototype is shown to two organisations: Telia Parken and Alarm HS. This resulted in two interviews aimed at getting a better understanding of use cases for the system, and feedback on the developed prototype.

The second sprint starts off with a revision of the system requirements, based on the two interviews in the preceding sprint. The interviews provided additional use cases, which were prioritised. We then explain some design considerations of the system, notably the platform of the system, a more thoroughly design architecture and GUI. As the system concerns itself with location data, the domain of geodetics is examined. At the end of the sprint, we met with Alarm HS again and also with a security officer from SmukFest. The security officer did not see a use for the prototype presented to him. Instead, he suggested a possible new direction for the system; A crowd simulator for events. We ultimately did not pursue this option, and instead continued the collaboration with Alarm HS.

The third sprint focuses on the technical side of determining the crowd factors described in sprint 1. The sprint starts with another revision of the requirements specification based on the meetings with Alarm HS and SmukFest. Afterwards, we introduce the kernel density estimation method, which we use to estimate crowd factors. The design of the analyser determining the crowd factors are then described, as well as its implementation as an algorithm.

Sprint four \sinote{fortsæt}