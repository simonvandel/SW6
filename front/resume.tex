\chapter*{Resume}
%% Resume af rapporten (1-2 sider)

This report concerns itself with the domain of crowd safety. Specifically, the report documents the development of an application for analysing crowd factors, based on location data. This location data is gathered using the aSTEP platform. The application is targeted towards crowd safety personnel working at large events such as music festivals. The report is divided into four sprints, each with a different focus.

In the first sprint, initial requirements for the system are based on assumptions about the problem domain. These requirements leads us to the concept of crowd conditions, which describes the state of a crowd in the context of crowd safety. Inspired by related work, four crowd factors are found to be of interest to this project: Density, velocity, turbulence and pressure. In order to be able to showcase our work to interested parties early on in the project, we develop a prototype. The primary focus of this prototype is on exploring functionality and usability of the system. This prototype is shown to two organisations, Telia Parken and Alarm HS, and the resulting interviews was aimed at getting a better understanding of use cases for the system, and feedback on the developed prototype.

The second sprint starts off with a revision of the system requirements, based on the two interviews in the preceding sprint. The interviews provided additional use cases, which were prioritised. Afterwards we explain some design considerations of the system, notably the platform choice, the architecture design, and the graphical user interface. As the system concerns itself with location data, the domain of geodetics is examined. At the end of the sprint, we met with Alarm HS again and also with a security officer from SmukFest. The security officer provided valuable feedback but did not see a use for the prototype, in his line of work.

The third sprint focuses on the technical side of determining the crowd factors described in the first sprint. The sprint starts with another revision of the requirements specification based on the meetings with Alarm HS and SmukFest. Afterwards, we introduce the kernel density estimation method, which we use to estimate crowd factors. The design of the analyser determining the crowd factors is then described, as well as its implementation as an algorithm.

The focus in the fourth and final sprint is on dealing with some non-functional requirements and preparing the project for a possible handover to other students. The sprint continues with the same requirements as in the previous sprint. The implementation of the class architecture is described in order to ease the work for other students. The sprint also describes the design of overlays visualising the four crowd factors on the front-end of the application. All design considerations are examined and reasoned about. At the end of the sprint, a final meeting is held with Alarm HS. The purpose of the meeting is to evaluate the system in a user acceptance test. The results of the test shows that while the system is usable, some issues remain.

The last chapters outlines the collaboration across all groups in the aSTEP multi-project, discusses the ethics and usefulness of the system, concludes on the project, and provides a list of future work items suitable for further development.