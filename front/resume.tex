\chapter*{Resume}
%% Resume af rapporten (1-2 sider)

This report concerns itself with the domain of crowd safety. Specifically, the report documents the development of a software system for analysing crowd conditions, based on location data on users. This location data is gathered using the aSTEP platform. The system is targeted for crowd safety personnel working at large events such as music festivals. The report is divided into four sprints, each with a different focus.

In the first sprint, initial requirements for the system are based on assumptions about the problem domain. These requirements leads us to the concept of crowd conditions, which describes the state of a crowd in the context of crowd safety. Inspired by prior work, four crowd factors are found to be of interest to this project: Density, velocity, turbulence and pressure. In order to be able to showcase our work to interested parties early on in the project, we develop a prototype. The primary focus of this prototype is on exploring functionalities and usability of the system. This prototype is shown to two organisations: Telia Parken and Alarm HS. This resulted in two interviews aimed at getting a better understanding of use cases for the system, and feedback on the developed prototype.

The second sprint starts off with a revision of the system requirements, based on the two interviews in the preceding sprint. The interviews provided many additional use cases, which were prioritised.